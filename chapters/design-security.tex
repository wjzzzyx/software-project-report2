\chapter{安全保密设计}
可能的内容包括保密性、是否采取加密传输、密钥如何分发和管理等。

安全性设计:

1、身份验证:用户登录系统才能进行操作

2、数据限制:访问数据库用户的所属类别决定访问的数据范围及操作权限

3、功能限制:通过用户功能视图限制用户对数据的操作

考生用户与系统之间传输的信息涉及到用户的隐私信息,应当保密,因此采用加密传输。使用https协议实现加密功能。每个考生用户的可见数据范围只有与自身相关的数据,这可以通过如下方式实现:在考生用户的每个请求中附加该用户的ID,把该ID作为查询条件的一部分。

ETS管理员用户直接在服务器端操作,不存在网络传输问题。应为各级管理人员设立相应的读写权限,这通过由DBA授权实现。