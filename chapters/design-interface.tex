\chapter{接口设计}
\section{外部接口}
使用支付宝接口。

\subsection{支付宝接口}
1、状态查询:EMS系统向支付宝接口发送状态查询请求,具体询问某一支付宝账户是否有效,若支付宝接口回应该账户有效,则可以进行下一步操作,某则EMS系统要求用户选择其他支付方式

2、支付交易:EMS系统向支付宝接口发送交易请求,此处EMS系统将明确交易金额,并告知支付宝接口收款对象,然后交由支付宝接口处理剩余交易操作,之后,等待支付宝接口的回复。若支付成功,则支付宝回复交易成功讯息,否则,支付宝回复交易失败讯息。

3、获取回执:EMS系统向支付宝接口发送回执请求,支付宝接口将把交易的详细信息通过表单的方式提交给EMS系统,EMS系统将表单数据录入数据库,作为用户信息的一部分

\subsection{微信、银联等接口}
同样具有状态查询、交易支付、获取回执等功能,其细节完全类似于支付宝,不再赘述

\subsection{其他接口}
由于ETS只有缴费环节涉及到与其他接口的交互,故不再需要除了支付宝等以外的接口


\section{内部接口}
考生用户在前端提交给服务器的请求具有如下结构:

用户的ID、请求的类型、附加的信息(取决于请求的类型,可选,如查询具体的某一场考试的考位时需提供考试信息,报名考试时需提供时间与考点)。

服务器返回的结果具有如下结构:

操作成功与否、指向查询信息的指针(如何解读查询结果取决于请求的类型)、错误提示(可选)

ETS管理人员在服务器端提交给系统的请求具有如下结构:

管理员的权限、操作类型、附加信息(通常为一组sql语句)

系统返回的结果具有如下结构:

操作成功与否、错误提示(可选)