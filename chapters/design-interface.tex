\chapter{接口设计}
\section{外部接口}
付款部分主要对支付宝接口作出说明。

\subsection{支付宝接口}
利用支付宝接口进行收款主要分为以下几个步骤:

1、下单:调用alipay.trade.page.pay,发起支付请求;

2、用户输入用户名、支付密码;

3、登陆;

4、选择支付渠道、输入支付密码;

5、确认支付;

6、get请求returnUrl,返回同步返回参数。这里,对于支付是否成功可以有两种判断方式:(1)支付是否成功以异步通知为准。此时,网站将post请求notifyURl,接受返回的异步通知参数来确定交易是否成功。如果成功,则需要将对应的考生支付信息录入对应的数据库;(2)支付是否成功以查询接口返回为准。此时,网站将调用alipay.trade.query查看交易状态,并根据返回的交易信息来确定交易是否成功。如果成功,则需要将考生的支付信息录入对应的数据库。

7、如果用户在交易成功之后想要退款,则用户点击退款按钮之后,需要调用alipay.trade.refund来发起退款请求。退款请求是否成功将返回,也可以调用alipay.trade.fastpay.refund.query来查询退款交易是否成功。

对于上述部分提及的接口,具体说明如下:

1、支付接口:

(1)商户系统请求支付宝接口alipay.trade.page.pay,支付宝对商户请求参数进行校验,而后重定向至用户登录页面。

(2)用户确认支付后,支付宝get请求returnUrl(商户入参传入),返回同步返回参数。

(3)交易成功后,支付宝post请求notifyUrl(商户入参传入),返回异步通知参数。

(4)若由于网络等问题异步通知没有到达,商户可自行调用alipay.trade.query接口进行查询,根据查询接口获取交易以及支付信息(商户也可以直接调用查询接口,不需要依赖异步通知)。

2、退款接口:若用户或商户需要退款,商户可调用alipay.trade.refund接口进行退款,支付宝同步返回退款参数。 

3、退款查询接口:若退款接口由于网络等原因返回异常,商户可调用alipay.trade.fastpay.refund.query退款查询接口查询指定交易的退款信息。

另外,还有两个其他的常用接口也值得说明:

1、交易关闭接口:通常交易关闭是通过alipay.trade.page.pay中的超时时间来控制,支付宝也提供给商户一个手动关闭交易的接口alipay.trade.close。若用户一直未支付,商户可以调用该接口关闭指定交易。成功关闭交易后该交易不可支付。若交易实际已经支付,商户仍然来调用alipay.trade.close,会进行全额退款。此时这笔交易也不可再进行支付。

2、为方便商户快速查账,支持商户通过alipay.data.dataservice.bill.downloadurl.query接口获取商户离线账单下载地址。

另外,对于各个接口的传入参数也有相关规定,例如对于alipay.trade.page.pay,对应的可能参数规定如下:

1、out\_trade\_no:类型为String的必填参数,最大长度为64。它代表一个商户订单号,在64个字符以内,可包含字母、数字、下划线;需保证在商户端不重复。

2、product\_code:类型为String的必填参数,最大长度为64。它代表销售产品码,即与支付宝签约的产品码名称。 注:目前仅支持
FAST\_INSTANT\_TRADE\_PAY。

3、total\_amount:类型为Price的必填参数,最大长度为11。它代表订单总金额,单位为元,精确到小数点后两位,取值范围[0.01,100000000]。

4、subject:类型为String的必填参数,最大长度为256。它代表订单标题。

5、body:类型为String的非必填参数,最大长度为128。它代表当前订单的描述。

6、goods\_type:类型为String的非必填参数,最大长度为2。它代表商品主类型,商品主类型:0—虚拟类商品,1—实物类商品(默认)。虚拟类商品不支持使用花呗渠道,对于ETS系统,此处应设置为0。

7、timeout\_express:类型为String的非必填参数,最大长度为6。它代表该笔订单允许的最晚付款时间,逾期将关闭交易。取值范围:1m~15d。m-分钟,h-小时,d-天,1c-当天(1c-当天的情况下,无论交易何时创建,都在0点关闭)。 该参数数值不接受小数点, 如 1.5h,可转换为90m。

该接口还支持一些其他参数,具体内容不一一列举,具体可以查阅支付宝开放平台提供的官方文档,以上内容中部分内容也直接援引自官方文档。其他接口(例如退款以及交易关闭接口)内参数的具体介绍同样不再一一列举。

\subsection{微信、银联等接口}
同样具有状态查询、交易支付、获取回执等功能,其细节完全类似于支付宝,不再赘述。

\subsection{其他接口}
由于ETS只有缴费环节涉及到与其他外部接口的交互,故不再需要除了支付宝等以外的外部接口。

\section{内部接口}
一下只对一些常用的,或内容较为复杂的接口做对应说明。一些内容简单的接口将不再一一列举。

\subsection{登录接口}
该接口接受考生/ETS管理人员的账号和密码信息,服务器将返回一个int类型的量来表示此次登录操作是否成功,如果成功则该值为0,否则该值将被设置为不同的值来代表对应的出错类型,例如:

1、返回值为1:代表用户的密码错误;

2、返回值为2:代表后端数据库的记录中不存在此用户;

3、返回值为3:代表用户名中包含非法的字符。

开发人员可以根据不同的值展示对应的出错信息。

\subsection{考试信息查询接口}
考生用户在前端提交给服务器的请求具有如下结构:

用户的ID、请求的类型、附加的信息(取决于请求的类型,可选,如查询具体的某一场考试的考位时需提供考试信息,报名考试时需提供时间与考点)。

服务器返回的结果具有如下结构:

操作成功与否、指向查询信息的指针(如何解读查询结果取决于请求的类型)、错误提示(可选)

ETS管理人员在服务器端提交给系统的请求具有如下结构:

管理员的权限、操作类型、附加信息(通常为一组sql语句)

系统返回的结果具有如下结构:

操作成功与否、错误提示(可选)

\subsection{题库更新接口}
ETS维护人员提交给服务器的请求具体包含如下参数:

1、更新的题目具体科目subject:类型为int的必填参数,由前端根据ETS管理人员的操作填写。0代表托福,1代表GRE。

2、更新题目所属区段的编号code:类型为int的必填参数。

3、更新题目的具体内容content:类型为String的必填参数,长度为10000。如果更新的内容属于多媒体内容,例如音频的话,那么此部分的类型可能需要重新指定。

返回的内容为一个bool型的变量,它表示更新结果是否成功的信息,如果成功则值为true,否则为false。

\subsection{考试结果录入接口}
该接口为考生考试系统同后端数据库的对接接口。具体而言,考生考试系统将向后端数据库发送以下参数作为一个合理的请求:

1、考生存储答题信息的表list:该表的每一行包括以下内容,即题目编号,题目类型和考生对应该题目的作答情况。如果题目类型为选择题,那么考生作答情况的区域值类型为int形,不同的整数值(如1、2、3、4)分别代表不同的选项。如果考生作答的为写作类题目,则作答情况区域的值为String类型,长度为5000。根据具体的题目,作答情况所对应的类型将有所不同,这个表将有前端自动处理生成。

2、考生的ID:唯一确定考生身份的标识符。

提交的对应内容将调用对应的接口写入后端数据库,这个接口同上述题库更新接口类似,不再具体说明。返回的内容为一个bool型的变量,它表示更新结果是否成功的信息,如果成功则值为true,否则为false。

\subsection{成绩生成接口}
ETS管理人员通过试卷批改系统调用此接口。这个接口将根据数据库内存储的信息完成相应题目的批改,并且将成绩信息写入对应的数据库中以便查询系统调用。调用此接口时将提交一下参数:

1、ETS管理人员的ID:用于确定该人员是否具有对应的权限。

2、需要批改的具体科目:类型为int的必填项。例如,0表示TOEFL,1表示GRE。

接受到该请求后,后端数据库将执行查询和比对操作,并且将对应的成绩存入成绩查询系统中。该接口将返回一个int值代表操作是否成功,具体如下:

1、值为0:表示操作成功;

2、值为1:表示操作所需的权限不够;

3、表示考生的考试信息未录入;

4、表示批改所需的答案未录入。
