\chapter{任务概述}
本系统的目标是实现一个ETS考试与管理系统,包括客户端、服务器端两个部分。

客户端面向考生用户,为用户提供账号注册、考位查询、缴费、考试报名、成绩查询等服务。
服务器端面向ETS管理员用户,为用户提供考试信息发布、题库维护、试题发布、试卷分发及测试、试卷批改等服务。

\section{目标}
实现ETS考试与管理系统,实现需求规格说明书中所描述的面向考生用户和面向ETS管理员用户的各个功能,并且保证系统的健壮性和数据安全。

\section{开发与运行环境}

\subsection{开发环境的配置}
\begin{table}[htbp]
\centering
\caption{开发环境的配置} \label{tab:development-environment}
\begin{tabular}{|c|c|c|}
    \hline
    类别 & 标准配置 & 最低配置 \\
    \hline
    计算机硬件 & \tabincell{c}{基于x86结构的CPU\\ 主频>=2.4GHz\\ 内存>=8G\\ 硬盘>=200G} & \tabincell{c}{基于x86结构的CPU\\ 主频>=1.6GHz\\ 内存>=512M\\ 硬盘>=2G} \\
    \hline
    计算机软件 & \tabincell{c}{Linux (kernel version>=4.10)\\ GNU gcc (version>=6.3.1)} & \tabincell{c}{Linux (kernel version>=3.10)\\ GNU gcc (version>=5.4)} \\
    \hline
    网络通信 & \tabincell{c}{至少要有一块可用网卡\\ 能运行IP协议栈即可} & \tabincell{c}{至少要有一块可用网卡\\ 能运行IP协议栈即可} \\
    \hline
    其他 & 采用Oracle数据库 & 采用Oracle数据库 \\
    \hline
\end{tabular}
% \note{这里是表的注释}
\end{table}

\subsection{测试环境的配置}
\begin{table}[htbp]
\centering
\caption{测试环境的配置} \label{tab:test-environment}
\begin{tabular}{|c|c|c|}
    \hline
    类别 & 标准配置 & 最低配置 \\
    \hline
    计算机硬件 & \tabincell{c}{基于x86结构的CPU\\ 主频>=2.4GHz\\ 内存>=8G\\ 硬盘>=200G} & \tabincell{c}{基于x86结构的CPU\\ 主频>=1.6GHz\\ 内存>=512M\\ 硬盘>=2G} \\
    \hline
    计算机软件 & \tabincell{c}{Linux (kernel version>=4.10)\\ GNU gcc (version>=6.3.1)} & \tabincell{c}{Linux (kernel version>=3.10)\\ GNU gcc (version>=5.4)} \\
    \hline
    网络通信 & \tabincell{c}{至少要有一块可用网卡\\ 能运行IP协议栈即可} & \tabincell{c}{至少要有一块可用网卡\\ 能运行IP协议栈即可} \\
    \hline
    其他 & 采用Oracle数据库 & 采用Oracle数据库 \\
    \hline

\end{tabular}
% \note{这里是表的注释}
\end{table}

\subsection{运行环境的配置}
\begin{table}[htbp]
\centering
\caption{运行环境的配置} \label{tab:operation-environment}
\begin{tabular}{|c|c|c|}
    \hline
    类别 & 标准配置 & 最低配置 \\
    \hline
    计算机硬件 & \tabincell{c}{基于x86结构的CPU\\ 主频>=2.4GHz\\ 内存>=16G\\ 硬盘>=16T} & \tabincell{c}{基于x86结构的CPU\\ 主频>=1.6GHz\\ 内存>=8G\\ 硬盘>=1T} \\
    \hline
    计算机软件 & \tabincell{c}{Linux (kernel version>=4.10)\\ GNU gcc (version>=6.3.1)} & \tabincell{c}{Linux (kernel version>=3.10)\\ GNU gcc (version>=5.4)} \\
    \hline
    网络通信 & \tabincell{c}{至少要有一块可用网卡\\ 能运行IP协议栈即可} & \tabincell{c}{至少要有一块可用网卡\\ 能运行IP协议栈即可} \\
    \hline
    其他 & 采用Oracle数据库 & 采用Oracle数据库 \\
    \hline

\end{tabular}
% \note{这里是表的注释}
\end{table}

\section{需求概述}
功能需求包括:

供考生使用的客户端系统提供的功能:新用户注册、个人信息管理、考试时间地点考位查询、考试报名、成绩查询。

供ETS管理员使用的服务器端系统提供的功能有:考试信息发布、题库维护、试题成型、试卷分发及测试、试卷批改。


\section{条件与限制}
本系统使用的技术约束如下:

编程语言:

1、前端:HTML+CSS+Javascript

2、后端:C++

接口:

1、操作系统:服务器端:Windows 2012 Server,客户端:支持浏览器的主流操作系统(Windows 7/8/8.1/10, Ubuntu等)
	
2、数据库:Oracle Database 12c 企业版
	
3、集成开发环境:Visual Studio 2015
	
4、库:ADO
	
5、通信:浏览器(HTTPS)
	
6、编程规范:参见《ETS考试报名、缴费与考场管理系统开发编程规范》中的具体说明。

