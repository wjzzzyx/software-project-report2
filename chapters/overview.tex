\chapter{任务概述}
本系统的目标是实现一个ETS考试与管理系统,包括客户端、服务器端两个部分。

客户端面向考生用户,为用户提供账号注册、考位查询、缴费、考试报名、成绩查询等服务。
服务器端面向ETS管理员用户,为用户提供考试信息发布、题库维护、试题发布、试卷分发及测试、试卷批改等服务。

\section{目标}
实现ETS考试与管理系统,实现需求规格说明书中所描述的面向考生用户和面向ETS管理员用户的各个功能,并且保证系统的健壮性和数据安全。

\section{开发与运行环境}

\subsection{开发环境的配置}
\begin{table}[htbp]
\centering
\caption{开发环境的配置} \label{tab:development-environment}
\begin{tabular}{|c|c|c|}
    \hline
    类别 & 标准配置 & 最低配置 \\
    \hline
    计算机硬件 & \tabincell{c}{因特尔 Xeon 5335 2颗\\ 主频 2.0GHz\\ 2G FBD内存 4条\\ 146GSAS 15000转硬盘 2 RAID 1级别容错} & \tabincell{c}{因特尔 Xeon 5405 1颗\\ 主频 2.0GHz\\ 1G FBD667内存 2条\\ 250G SATA硬盘 2 RAID 1级别容错} \\
    \hline
    计算机软件 & \tabincell{c}{微软 Windows Server 2016\\ Microsoft Visual C++ 2015} & \tabincell{c}{微软 Windows Server 2016\\ Microsoft Visual C++ 2015} \\
    \hline
    网络通信 & \tabincell{c}{思科WMP600N 300m 双频 PCI无线网卡\\ 能运行IP协议栈} & \tabincell{c}{思科AE1200 AE2500双频USB无线网卡\\ 能运行IP协议栈} \\
    \hline
    数据库 & Oracle Database 12c 企业版 & Oracle Database 12c 企业版 \\
    \hline
\end{tabular}
% \note{这里是表的注释}
\end{table}

\subsection{测试环境的配置}
\begin{table}[htbp]
\centering
\caption{测试环境的配置} \label{tab:test-environment}
\begin{tabular}{|c|c|c|}
    \hline
    类别 & 标准配置 & 最低配置 \\
    \hline
    计算机硬件 & \tabincell{c}{因特尔 Xeon 5335 2颗\\ 主频 2.0GHz\\ 2G FBD内存 4条\\ 146GSAS 15000转硬盘 2 RAID 1级别容错} & \tabincell{c}{因特尔 Xeon 5405 1颗\\ 主频 2.0GHz\\ 1G FBD667内存 2条\\ 250G SATA硬盘 2 RAID 1级别容错} \\
    \hline
    计算机软件 & \tabincell{c}{微软 Windows Server 2016\\ Microsoft Visual C++ 2015} & \tabincell{c}{微软 Windows Server 2016\\ Microsoft Visual C++ 2015} \\
    \hline
    网络通信 & \tabincell{c}{思科WMP600N 300m 双频 PCI无线网卡\\ 能运行IP协议栈} & \tabincell{c}{思科AE1200 AE2500双频USB无线网卡\\ 能运行IP协议栈} \\
    \hline
    数据库 & Oracle Database 12c 企业版 & Oracle Database 12c 企业版 \\
    \hline

\end{tabular}
% \note{这里是表的注释}
\end{table}

\subsection{运行环境的配置}
\begin{table}[htbp]
\centering
\caption{运行环境的配置} \label{tab:operation-environment}
\begin{tabular}{|c|c|c|}
    \hline
    类别 & 标准配置 & 最低配置 \\
    \hline
    计算机硬件 & \tabincell{c}{因特尔 Xeon 5335 2颗\\ 主频 2.0GHz\\ 2G FBD内存 4条\\ 146GSAS 15000转硬盘 2 RAID 1级别容错} & \tabincell{c}{因特尔 Xeon 5405 1颗\\ 主频 2.0GHz\\ 1G FBD667内存 2条\\ 250G SATA硬盘 2 RAID 1级别容错} \\
    \hline
    计算机软件 & \tabincell{c}{微软 Windows Server 2016\\ Microsoft Visual C++ 2015} & \tabincell{c}{微软 Windows Server 2016\\ Microsoft Visual C++ 2015} \\
    \hline
    网络通信 & \tabincell{c}{思科WMP600N 300m 双频 PCI无线网卡\\ 能运行IP协议栈} & \tabincell{c}{思科AE1200 AE2500双频USB无线网卡\\ 能运行IP协议栈} \\
    \hline
    数据库 & Oracle Database 12c 企业版 & Oracle Database 12c 企业版 \\
    \hline

\end{tabular}
% \note{这里是表的注释}
\end{table}

\section{需求概述}
功能需求包括:

供考生使用的客户端系统提供的功能:新用户注册、个人信息管理、考试时间地点考位查询、考试报名、成绩查询。

供ETS管理员使用的服务器端系统提供的功能有:考试信息发布、题库维护、试题成型、试卷分发及测试、试卷批改。


\section{条件与限制}
本系统使用的技术约束如下:

编程语言:

1、前端:HTML+CSS+Javascript

2、后端:C++

接口:

1、操作系统:服务器端:Windows Server 2016,客户端:支持浏览器的主流操作系统(Windows 7/8/8.1/10, Ubuntu等)
	
2、数据库:Oracle Database 12c 企业版
	
3、集成开发环境:Visual Studio 2015
	
4、库:ADO
	
5、通信:浏览器(HTTPS)
	
6、编程规范:参见《ETS考试报名、缴费与考场管理系统开发编程规范》中的具体说明。

