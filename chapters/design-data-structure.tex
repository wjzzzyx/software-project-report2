\chapter{数据结构设计}
\section{逻辑结构设计}
\subsection{数据结构设计}
本章节主要描述程序运行逻辑中,除了数据库部分外,还需要额外使用的数据结构。

由于ETS考试管理系统的本质为客户端—服务器系统,其中客户端为用户进行注册、登录、信息查询、更新等操作的系统,服务器端为管理员进行信息发布、试题添加、试题成型、阅卷评分的操作的系统,故本章在数据结构的描述上将分客户端数据结构与服务器端数据结构进行。

\subsection{客户端数据结构}
1、存储用户个人信息的结构体,其域包括:用户名、密码、真实姓名、生日、身份证号、电子邮箱、联系电话、联系地址等,通过该结构体,客户端能及时记录下用户注册所填写的个人信息,之后再通过表单的形式发送给服务器端,从而更新服务器端的数据库

2、存储用户查询信息的结构体,其域包括:查询指令、查询参数等,通过该结构体,客户端能及时记录下用户希望查询的信息,之后可通过表单的形式发送给服务器端,从而获取服务器端的响应

3、存储用户试题作答情况的索引数组,其存储用户作答时每个答案存储的地址,在提交答案时,需要将该索引数组与存储的所有答案一并提交给服务器端,服务器端通过对索引数组的解析,从答案集中分离出各个题目的答案,然后存入数据库中,留待阅卷评分阶段对其进行操作

\subsection{服务器端数据结构}
1、大量数据库表,存储系统运行所必须的信息

2、优先队列,实现对所有学生成绩的动态排序,从而能够生成考生考试情况的位次信息,位次信息作为反馈信息的一部分,能够用来宏观调控最终的考生成绩,从而保证在不同试题难度下,分数能够做到相对客观

3、普通队列,存储所有请求,从而保证能够依次对各个请求进行处理

\section{物理结构设计}
各数据结构无特殊物理结构要求。

\section{数据结构与程序模块的关系}
[此处指的是不同的数据结构分配到哪些模块去实现。可按不同的端拆分此表]
\begin{table}[htbp]
\centering
\caption{数据结构与程序代码的关系表} \label{tab:datastructure-module}
\begin{tabular}{|c|c|c|}
    \hline
    · & 考生用户子系统 & ETS管理子系统 \\
    \hline
    用户信息结构体 & Y & · \\
    \hline
    考生用户查询信息结构体 & Y & · \\
    \hline
    ETS管理员查询信息结构体 & · & Y \\
    \hline
    试题索引数组 & · & Y \\
    \hline
    优先队列 & · & Y \\
    \hline
    普通队列 & Y & · \\
    \hline
\end{tabular}
\note{各项数据结构的实现与各个程序模块的分配关系}
\end{table}