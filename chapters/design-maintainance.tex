\chapter{维护设计}

\section{软件维护}
\subsection{改正性维护}
软件运行过程中一旦发现错误,用户会向开发人员提出改正性维护的请求。通常遇到的错误有:设计错误、逻辑错误、编码错误、文档、数据错误。

\subsection{适应性维护}
软件运行硬件、OS、网络等环境发生变化后,要适应环境变化和发展而对软件进行维护。环境变化主要包括:影响系统的规定、法律和规则发生了变化;硬件配置发生了变化,例如机型、终端等;数据格式和文件结构发生变化;系统软件发生变化,例如操作系统、编译系统、实用软件包的变化。

\subsection{完善性维护}
当用户要求增加新的功能、修改已有功能,或提出其他改进意见时,为满足用户日益增长的各种需求而对软件系统进行的改善和补充。

\subsection{预防性维护}
改进软件质量属性,为以后的软件改进奠定基础的维护活动。

\section{数据库维护}
可能的内容包括数据库的日常备份、压缩、维护等。

数据库中存储的历史信息十分重要,因此需定期备份,并永久存储。

数据库中的数据按重要性和及时性分类,不同的数据需要不同的备份频次。例如,考生信息数据会有频繁的增删改操作,因此需有较高的备份频率,如每小时备份一次。为了压缩总存储信息量,一定时间段(例如24小时)前的备份将被新的备份覆盖。而未来12月的考试信息数据不会经常变更,因此可以一个月备份一次。当前月份之前的考试信息都可以丢弃。

除了备份以外,可能的事件会触发数据的迁移。例如,当一次考试举行后,考试注册数据表内的该次考试的注册信息就不再有用,应当转移到成绩数据表中。凡是参加了这次考试的考生将在成绩数据表中获得一个对应条目,附加上本次考试的成绩。这个工作可以由ETS管理人员完成,也可以交给程序自动完成。

ETS管理人员还需维护题库,并相应地维护试卷数据表中的信息。当有一套试题成型时,就将它添加到试卷数据表中。