\chapter{引言}
\section{编写目的}
在本项目的前一阶段,也就是需求分析阶段,已经将系统用户对本系统的需求做了详细的阐述,这些用户需求已经在上一阶段中对不同用户所提出的不同功能,实现的各种效果做了调研工作,并在需求规格说明书中得到详尽得叙述及阐明。

本阶段已在系统的需求分析的基础上,对ETS考试与管理系统做概要设计。主要解决了实现该系统需求的程序模块设计问题。包括如何把该系统划分成若干个模块、决定各个模块之间的接口、模块之间传递的信息,以及数据结构、模块结构的设计等。在以下的概要设计报告中将对在本阶段中对系统所做的所有概要设计进行详细的说明,在设计过程中起到了提纲挈领的作用。

在下一阶段的详细设计中,程序设计员可参考此概要设计报告,在概要设计即时聊天工具所做的模块结构设计的基础上,对系统进行详细设计。在以后的软件测试以及软件维护阶段也可参考此说明书,以便于了解在概要设计过程中所完成的各模块设计结构,或在修改时找出在本阶段设计的不足或错误。


\section{项目背景}
随着我国教育事业的不断发展,高等教育水平的不断提高,越来越多的学子选择海外留学。美国作为世界头号强国,自然成为了众多学子留学的首选目标。而要获得赴美留学的资格,通过ETS指定的TOFEL、GRE、GMAT考试是必经之路。为了方便考生参加这些考试,为了使考试信息化、网络化,也为了ETS管理的方便,特开发了这套ETS考试与管理系统。

\section{术语}
[列出本文档中所用到的专门术语的定义和外文缩写的原词组]
\begin{table}[htbp]
\centering
\caption{术语表} \label{tab:terminology}
\begin{tabular}{|c|c|}
    \hline
    缩写、术语 & 解释 \\
    \hline
    考生 & 使用本系统进行考试报名、成绩查询等操作的个人 \\
    \hline
    ETS管理员 & 隶属于ETS,使用本系统进行数据维护与信息发布的人员 \\
    \hline
    表单 & 在网络上传送消息的数据结构 \\
    \hline
    题库维护 & 对所有候选试题的增删改查 \\
    \hline
    试题成型 & 选定一系列试题组合为一份试卷 \\
    \hline
    数据库完整性 & 数据满足设定的约束的性质 \\
    \hline
\end{tabular}
% \note{这里是表的注释}
\end{table}